\input{style/sqtbeamer.tex}

\title{Quant 101}
\subtitle{Introduction to Quantitative Finance}
\institute{Sydney Quant Traders}
\date{}

\begin{document}

\maketitle

\section{Markets \& Exchanges}

\begin{frame}{Trades}
  \begin{sqt:definition}
    A \textbf{trade} is swapping money and assets
  \end{sqt:definition}

  \begin{sqt:definition}
    \textbf{Settlement} is when ownership of the asset and payment are actually transferred between parties.
  \end{sqt:definition}

  \vspace{0.3cm}
  Settlement conventions (where T = trade date, counting in business days):
  \begin{itemize}
    \item \textbf{T+0} (`spot', e.g. blockchain)
    \item \textbf{T+1, T+2, \dots} (`clearing', e.g. equities)
    \item \textbf{March 20th} (`futures')
  \end{itemize}
\end{frame}

\begin{frame}{What is an Exchange}
  \begin{sqt:definition}
    A \textbf{securities exchange} is a place where people gather and express their desire to buy or sell.
    The exchange matches participants to form \textbf{trades}.
  \end{sqt:definition}
  \vspace{0.5cm}

  \begin{columns}
    \begin{column}[t]{0.5\textwidth}
      US exchanges
      \begin{itemize}
        \item (XNYS) NYSE
        \item (XNAS) Nasdaq
        \item (XCBO) CBOE
        \item (XCME) CME
      \end{itemize}
    \end{column}

    \begin{column}[t]{0.5\textwidth}
      APAC exchanges
      \begin{itemize}
        \item (XHKG) HKEX
        \item (XSHG) SSE
        \item (XSGE) SHFE
      \end{itemize}
    \end{column}
  \end{columns}
\end{frame}

\begin{frame}{SKIP}
  TODO: explain exchanges more after watching Oden's video
\end{frame}

\begin{frame}{Limit Orderbook}
  % TODO: note that this is just one mechanism to do counterparty discovery
  Participants express their desire to buy or sell a particular instrument by maintaining a set of \textbf{limit orders} on the exchange.

  Each order represents an intention to buy or sell a specified quantity of an instrument, where each instrument is traded at a price within some specified price range.

  TODO: finish this slide after watching Oden's video
\end{frame}

\begin{frame}{Limit Orderbook Examples}
  \begin{columns}[T]
    \begin{column}{0.4\textwidth}
      \begin{tabular}{|c|c|c|}
        \hline
        \textbf{Bid Qty}        & \textbf{Price}          & \textbf{Ask Qty}       \\
        \hline
        \cellcolor{gray!10}     & \cellcolor{gray!10}130  & \cellcolor{gray!10}24  \\
        \cellcolor{gray!10}     & \cellcolor{gray!10}125  & \cellcolor{gray!10}20  \\
        \cellcolor{gray!10}     & \cellcolor{gray!10}120  & \cellcolor{gray!10}10  \\
        \cellcolor{gray!10}15   & \cellcolor{gray!10}80   & \cellcolor{gray!10}    \\
        \cellcolor{gray!10}15   & \cellcolor{gray!10}75   & \cellcolor{gray!10}    \\
        \cellcolor{gray!10}30   & \cellcolor{gray!10}70   & \cellcolor{gray!10}    \\
        \hline
      \end{tabular}
    \end{column}

    \begin{column}{0.6\textwidth}
      \textbf{Discussion Questions}
      \begin{itemize}
        \item Why are all bids prices below all ask prices?
        \item Why do the quantities of quotes decrease as we approach the middle price?
        \item Would we consider this orderbook liquid?
        \item What is the market price of AMZN in this orderbook?
      \end{itemize}
    \end{column}
  \end{columns}

\end{frame}

\begin{frame}{The Exchange Business Model}
  Exchanges make money by charging a fee per trade.

  This fee is usually proportional to the dollar volume of the trade.

  Exchanges benefit from high trading volume, so they have an incentive to attract participants to their platform.

  To attract customers, exchanges do the following:
  \begin{itemize}
    \item They provide a fair and transparent trading environment.
    \item Have high liquidity.
  \end{itemize}
\end{frame}

\begin{frame}{Market Markets}
  \begin{sqt:definition}
    A \textbf{market maker} is a participant in an exchange who provides liquidity by continuously quoting buy and sell prices for a security.
  \end{sqt:definition}

  Market makers must fulfill obligations to the exchange, or lose their status as a market maker.
  \begin{itemize}
    \item Quote a minimum size at all times.
    \item Quote within a certain spread.
  \end{itemize}
\end{frame}

\section{Orderbook Examples}

\section{Trading \& Research}

\begin{frame}{Trading and Research Roles}
  \begin{itemize}
    \item \textbf{Trader}
    \item \textbf{Researcher}
  \end{itemize}

  \vspace{0.5cm}
  \begin{sqt:result}
    The boundaries between these two roles may across firms and teams.
  \end{sqt:result}
\end{frame}

\begin{frame}{Trader}
  TODO: get help for this slide
\end{frame}

\begin{frame}{Researcher}
  There is much more to research than just alpha and pricing theory.

  There are a wide variety of research roles, but they generally fall into the following categories:

  \begin{itemize}
    \item \textbf{Alpha}: Discovering predictive signals and trading opportunities
    \item \textbf{Pricing}: Pricing exotics (e.g. barrier options)
    \item \textbf{Risk}: Risk
    \item \textbf{Execution}: Transaction costs analysis
    \item \textbf{Portfolio Optimization}: 
  \end{itemize}

  Roles also look different depending on time-horizon.

  TODO: finish this slide later
\end{frame}

\begin{frame}{Researcher Interviews: Core Skills}
  These interviews also have probability questions.
  Also may have questions about statistics or take home projects.
\end{frame}

\section{Engineering}

\begin{frame}{Engineering Roles}

  \begin{itemize}
    \item \textbf{Software Engineer}
    \item \textbf{Quantitative Developer}
    \item \textbf{Hardware Engineer}
  \end{itemize}
\end{frame}

\begin{frame}{Software Engineering}
  In trading firms, software engineers do much of the same work as software engineers in other industries:
  \begin{itemize}
    \item \textbf{Application Development}
    \item \textbf{UI Development}
    \item \textbf{Data Engineering}
  \end{itemize}

  \begin{sqt:result}
    There isn't really a difference between the software roles.
  \end{sqt:result}
\end{frame}

\begin{frame}{Software Engineering Interviews: Core Skills}
  Fundamental competencies expected across all software roles during the interview

  \begin{itemize}
    \item \textbf{Data Structures \& Algorithms}: binary search, graphs, complexity analysis
    \item \textbf{Design Skills}: System design, API design, code design, OOP
    \item \textbf{General Knowledge}: Databases, Networks, OS, Concurrency, Compilers
    \item \textbf{Testing}: Unit tests, mock objects, designing for testability
  \end{itemize}
\end{frame}

\begin{frame}{Software Engineering Interviews: Specialized Tracks}
  Many firms separate candidates into specialized tracks early in the process.

  \vspace{0.3cm}
  \textbf{Low-Latency Track}
  \begin{itemize}
    \item C++
    \item Operating Systems
    \item Computer Architecture
  \end{itemize}

  \vspace{0.3cm}
  \textit{Example firms: Citadel Securities, QRT}
\end{frame}

\begin{frame}{Software Engineering Interviews: Specialized Tracks}
  \textbf{Python Track}
  \begin{itemize}
    \item Probability and statistics
    \item Data manipulation (NumPy, Pandas, SQL)
    \item Other stats (machine learning, data science)
  \end{itemize}

  \textit{Example firms: Citadel Securities}
\end{frame}

\begin{frame}{Quantitative Development}
  Half developer, half researcher.

  \vspace{0.3cm}
  \textbf{Math/Stats Skills}
  \begin{itemize}
    \item Undergrad math (linear algebra, probability, statistics)
    \item ML techniques (linear regression, clustering, decision trees, gradient boosting)
    \item Python for research (NumPy, Pandas, sklearn, statsmodels, plotting libraries)
  \end{itemize}

  \vspace{0.3cm}
  \textbf{Dev Skills}
  \begin{itemize}
    \item Expected to be as proficient as any other dev
  \end{itemize}

  \vspace{0.3cm}
  \textit{Example firms: Citadel Securities}
\end{frame}

\section{Resources}

\begin{frame}{Resources}
  \textbf{Talks}
  \begin{itemize}
    \item \href{https://www.youtube.com/watch?v=hYmR5qk4z8w}{Oden Petersen \& QuantSoc - Systematic Trading Workshop}
    \item \href{https://www.youtube.com/watch?v=2YkK4DnEBRE}{George Mukun Wang \& SQT - Intro to Quant Trading}
  \end{itemize}
\end{frame}

\begin{frame}{Resources}
  \textbf{Software Engineering Classics}
  \begin{itemize}
    \item \textit{Introduction to Algorithms} (CLRS)
    \item \textit{Designing Data-Intensive Applications} (DDIA)
    \item \textit{Operating Systems: Three Easy Pieces} (OSTEP)
    \item \textit{Exceptional C++} by Herb Sutter
    \item \textit{Effective Modern C++} by Scott Meyers
  \end{itemize}
\end{frame}

\end{document}